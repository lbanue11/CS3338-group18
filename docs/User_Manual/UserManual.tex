\documentclass[11pt]{article}

\usepackage[margin=1in]{geometry}
\usepackage[hidelinks]{hyperref}
\usepackage{setspace}
\setstretch{1.15}

\title{SUAS Flight Path Visualization\\[0.5em]\large User Manual}
\author{
  Leo Banuelos \and
  Eddie Echegoyen \and
  Jeffrey Li \and
  Kent Stark
}
\date{\today}

\begin{document}
\maketitle

\section{Introduction}

Small, uncrewed aircraft systems and drones can provide small military units
with valuable intelligence, surveillance, and reconnaissance information.
However, the flight paths of these drones must consider several factors to
successfully gather the needed information while minimizing the chances of
being detected. These factors may include models of the drone's noise and
visibility signatures, models of human vision and auditory perception,
environmental factors, and models of drone sensors, among others.

The goal of this project is to give an operator a simple way to plan and
visualize drone flight paths over a map and 3D terrain. The operator can see
how the drone will move, adjust the path, and try to meet mission goals while
reducing the chance of detection.

\section{Project Objective}

The objective of this project is to develop, test, and deploy a 3D
visualization architecture that lets operators:
\begin{itemize}
  \item Plan drone flight paths over a selected terrain.
  \item Visualize the path in 2D and 3D.
  \item Modify the path based on mission needs.
  \item Consider detection risk and sensor coverage while planning.
\end{itemize}

\section{Project Links}

\textbf{Jira Project Link:}\\
\href{https://cs3338-group-18.atlassian.net/jira/software/projects/CG1S/boards/4}{https://cs3338-group-18.atlassian.net/jira/software/projects/CG1S/boards/4}

\vspace{0.5em}

\textbf{GitHub Repository:}\\
\href{https://github.com/lbanue11/CS3338-group18}{https://github.com/lbanue11/CS3338-group18}

\end{document}
