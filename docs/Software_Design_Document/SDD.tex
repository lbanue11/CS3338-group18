\documentclass[11pt]{article}

\usepackage[margin=1in]{geometry}
\usepackage[hidelinks]{hyperref}
\usepackage{setspace}
\usepackage{array}
\setstretch{1.15}

\title{Software Design Document\\[0.5em]
SUAS Flight Path Visualization}
\author{
  Leo Banuelos \and
  Eddie Echegoyen \and
  Jeffrey Li \and
  Kent Stark
}


\begin{document}
\maketitle

\section*{Revision History}
\begin{tabular}{|>{\raggedright}p{2cm}|p{3cm}|p{8cm}|}
\hline
\textbf{Version} & \textbf{Date} & \textbf{Description} \\
\hline
1.0 & Snapshot 1 & Initial draft. \\
\hline
2.0 & Snapshot 2 & Added frontend / backend \\
\hline
\end{tabular}

\section{Introduction}

\subsection{Purpose}
This Software Design Document (SDD) describes the main design decisions for
the SUAS Flight Path Visualization project. It explains how the system is
structured so future developers can understand and extend it.

\subsection{Scope}
The system is a web application that helps operators plan, visualize, and
adjust drone flight paths over a map and 3D terrain while keeping detection
risk in mind.

\subsection{References}
\begin{itemize}
  \item Software Requirements Specification (SRS) for SUAS Flight Path Visualization.
  \item User Manual for SUAS Flight Path Visualization.
  \item Jira Project: \href{https://cs3338-group-18.atlassian.net/jira/software/projects/CG1S/boards/4}{CG1S Board}.
  \item GitHub Repository: \href{https://github.com/lbanue11/CS3338-group18}{CS3338-group18}.
\end{itemize}

\section{System Overview}

At a high level, the system has three main parts:
\begin{itemize}
  \item \textbf{Frontend Web UI} -- Runs in the browser, shows the map and 3D
        view, lets the user draw and edit flight paths.
  \item \textbf{Backend Service} -- Exposes APIs to save and load missions,
        compute simple path metrics, and prepare data for visualization.
  \item \textbf{Data Layer} -- Stores user missions, terrain references, and
        drone / sensor presets.
\end{itemize}

The frontend talks to the backend over HTTP using JSON. The backend reads and
writes data in the storage layer and returns results to the UI.

\section{Architectural Design}

\subsection{Frontend (Client)}

\begin{itemize}
  \item Single–page application rendered in the browser.
  \item Uses a map/3D library (e.g., Mapbox and/or three.js) to show terrain and
        paths.
  \item Handles user input: adding, moving, and deleting waypoints; switching
        between 2D and 3D views; starting simulations.
  \item Calls backend REST endpoints to load and save missions.
\end{itemize}

\subsection{Backend (Server)}

\begin{itemize}
  \item Simple web API (e.g., Node.js/Express).
  \item Endpoints for:
        \begin{itemize}
          \item Creating and updating flight paths.
          \item Listing and loading saved missions.
          \item Returning basic analytics (total distance, estimated duration,
                simple risk score).
        \end{itemize}
  \item Performs input validation and basic business logic.
\end{itemize}

\subsection{Data Storage}

\begin{itemize}
  \item Stores missions as documents or rows with:
        \begin{itemize}
          \item Mission metadata (name, date, owner).
          \item List of waypoints (lat, lon, altitude).
          \item Selected drone type and terrain scenario.
        \end{itemize}
  \item Storage can be a lightweight database or JSON files, depending on
        the deployment.
\end{itemize}

\end{document}
