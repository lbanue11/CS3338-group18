\documentclass[11pt]{article}

\usepackage[margin=1in]{geometry}
\usepackage[hidelinks]{hyperref}
\usepackage{setspace}
\usepackage{array}
\setstretch{1.15}

\title{Software Requirements Specification\\[0.5em]
SUAS Flight Path Visualization}
\author{
  Leo Banuelos \and
  Eddie Echegoyen \and
  Jeffrey Li \and
  Kent Stark
}

\begin{document}
\maketitle

\section*{Revision History}
\begin{tabular}{|>{\raggedright}p{2cm}|p{3cm}|p{8cm}|}
\hline
\textbf{Version} & \textbf{Date} & \textbf{Description} \\
\hline
1.0 & Snapshot 1 & First draft of requirements and basic scope. \\
\hline
\end{tabular}

\section{Introduction}

This document describes what the SUAS Flight Path Visualization project is
supposed to do. It is a simple list of features and expectations so that the
design and code match the basic idea of the project.

The project is a web-based tool to help plan and visualize drone flight paths
over terrain, with a focus on keeping detection risk in mind.

\subsection*{Links}
Jira Board: \\
\href{https://cs3338-group-18.atlassian.net/jira/software/projects/CG1S/boards/4}{https://cs3338-group-18.atlassian.net/jira/software/projects/CG1S/boards/4}

\vspace{0.5em}

GitHub Repository: \\
\href{https://github.com/lbanue11/CS3338-group18}{https://github.com/lbanue11/CS3338-group18}

\section{Overall Description}

\subsection*{What the System Is}

The system is a simple planning tool that runs in a web browser. A user can see
a map or basic 3D terrain, draw a path for a drone, and make changes until the
path looks good for the mission.

\section{Functional Requirements}

The system should be able to do at least the following:

\begin{itemize}
  \item Let the user create a flight path by clicking on the map to add points.
  \item Let the user move or delete points on the path.
  \item Show the path on a 2D map.
\end{itemize}

These requirements are enough for a basic working demo. Extra features can be
added later if there is time.

\end{document}
