\documentclass[11pt]{article}

\usepackage[margin=1in]{geometry}
\usepackage[hidelinks]{hyperref}
\usepackage{setspace}
\usepackage{array}
\setstretch{1.15}

\title{Software Requirements Specification\\[0.5em]
SUAS Flight Path Visualization}
\author{
  Leo Banuelos \and
  Eddie Echegoyen \and
  Jeffrey Li \and
  Kent Stark
}


\begin{document}
\maketitle

\section*{Revision History}
\begin{tabular}{|>{\raggedright}p{2cm}|p{3cm}|p{8cm}|}
\hline
\textbf{Version} & \textbf{Date} & \textbf{Description} \\
\hline
1.0 & Snapshot 1 & First draft of requirements and basic scope. \\
\hline
2.0 & Snapshot 2 & Added more details for features and UI. \\
\hline
3.0 & Snapshot 3 & Cleaned up requirements and added a few notes. \\
\hline
4.0 & Snapshot 4 & Final edits \\
\hline
\end{tabular}

\section{Introduction}

This document describes what the SUAS Flight Path Visualization project is
supposed to do. It is a simple list of features and expectations so that the
design and code match the basic idea of the project.

The project is a web-based tool to help plan and visualize drone flight paths
over terrain, with a focus on keeping detection risk in mind.

\subsection*{Links}
Jira Board: \\
\href{https://cs3338-group-18.atlassian.net/jira/software/projects/CG1S/boards/4}{https://cs3338-group-18.atlassian.net/jira/software/projects/CG1S/boards/4}

\vspace{0.5em}

GitHub Repository: \\
\href{https://github.com/lbanue11/CS3338-group18}{https://github.com/lbanue11/CS3338-group18}

\section{Overall Description}

\subsection*{What the System Is}

The system is a simple planning tool that runs in a web browser. A user can see
a map or basic 3D terrain, draw a path for a drone, and make changes until the
path looks good for the mission.

\subsection*{Who Uses It}

\begin{itemize}
  \item Students and developers who are working on or demoing the project.
  \item Instructors or sponsors who want to see example flight paths.
  \item Anyone using it for basic mission planning practice.
\end{itemize}

\subsection*{Where It Runs}

\begin{itemize}
  \item Desktop or laptop computer.
  \item Modern web browser (Chrome, Firefox, Edge).
  \item Internet connection so maps and data can load.
\end{itemize}

\section{Functional Requirements}

The system should be able to do at least the following:

\begin{itemize}
  \item Let the user create a flight path by clicking on the map to add points.
  \item Let the user move or delete points on the path.
  \item Show the path on a 2D map.
  \item Provide a simple 3D view of the terrain and the path.
  \item Show basic info like total distance and an estimated flight time.
  \item Allow the user to save a mission (flight path with a name).
  \item Allow the user to load a previously saved mission.
  \item Let the user pick a drone type or preset for the mission.
  \item Show some kind of simple “risk” or “attention” indicator along the path
        (for example, color changes on certain segments).
\end{itemize}

These requirements are enough for a basic working demo. Extra features can be
added later if there is time.

\section{Non-Functional Requirements }

\subsection*{Performance}

\begin{itemize}
  \item The map should respond in a reasonable time when the user pans or zooms.
  \item Loading a saved mission should not take too long for normal-sized paths.
\end{itemize}

\subsection*{Usability}

\begin{itemize}
  \item New users should be able to figure out how to add and move points on
        the map with minimal instructions.
  \item The main controls (create, save, load, delete) should be easy to find.
\end{itemize}

\subsection*{Safety and Scope}

\begin{itemize}
  \item The system is only for planning and visualization.
  \item It does not control real drones or send real commands.
\end{itemize}

\subsection*{Security}

\begin{itemize}
  \item If user accounts are used, only logged-in users should see their own
        saved missions.
  \item Communication between browser and server should use normal secure
        web traffic (HTTPS) when possible.
\end{itemize}

\end{document}
